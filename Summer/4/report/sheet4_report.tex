% TEMPLATE.TEX
%
% Time-stamp: <2013-03-26 11:09 olenz>
%
% This is an extensively documented LaTeX file that shows how to
% produce a good-looking document with current LaTeX (11/2012).
%
% IMPORTANT!
%
%   Some obsolete commands and packages
% ----------|-------------------------------
% obsolete  |     Replacement in LATEX 2ε
% ----------|-------------------------------
%           | local            global/switch
% ----------|-------------------------------
% {\bf ...} | \textbf{...}     \bfseries
%     -     | \emph{...}       \em
% {\it ...} | \textit{...}     \itshape
%     -     | \textmd{...}     \mdseries
% {\rm ...} | \textrm{...}     \rmfamily
% {\sc ...} | \textsc{...}     \scshape
% {\sf ...} | \textsf{...}     \sffamily
% {\sl ...} | \textsl{...}     \slshape
% {\tt ...} | \texttt{...}     \ttfamily
%     -     | \textup{...}     \upshape
%
% DON'T USE \\ TO MAKE LINEBREAKS, INSTEAD JUST LEAVE A BLANK LINE!
%
\RequirePackage[l2tabu,orthodox]{nag} % turn on warnings because of bad style
\documentclass[a4paper,10pt,bibtotoc]{scrartcl}
%
\usepackage[bottom=3.5cm, top=2cm]{geometry}
%%%%%%%%%%%%%%%%%%%%%%%%%%%%%%%%%%%%
% KOMA CLASSES
%%%%%%%%%%%%%%%%%%%%%%%%%%%%%%%%%%%%
%
% The class "scrartcl" is one of the so-called KOMA-classes, a set of
% very well done LaTeX-classes that produce a very European layout
% (e.g. titles with a sans-serif font).
%
% The KOMA classes have extensive documentation that you can access
% via the commands:
%   texdoc scrguide # in German
%   texdoc scrguien # in English
%
%
% The available classes are:
%
% scrartcl - for "articles", typically for up to ~20 pages, the
%            highest level sectioning command is \section
%
% scrreprt - for "reports", typically for up to ~200 pages, the
%            highest level sectioning command is \chapter
%
% scrbook  - for "books", for more than 200 pages, the highest level
%            sectioning command is \part.
%
% USEFUL OPTIONS
%
% a4paper  - Use a4 paper instead of the default american letter
%            format.
%
% 11pt, 12pt, 10pt
%          - Use a font with the given size.
%
% bibtotoc - Add the bibliography to the table of contents
%
% The KOMA-script classes have plenty of options to modify

% This allows to type UTF-8 characters like ä,ö,ü,ß
\usepackage[utf8]{inputenc}

\usepackage[T1]{fontenc}        % Tries to use Postscript Type 1 Fonts for better rendering
\usepackage{lmodern}            % Provides the Latin Modern Font which offers more glyphs than the default Computer Modern
\usepackage[intlimits]{amsmath} % Provides all mathematical commands

\usepackage{hyperref}           % Provides clickable links in the PDF-document for \ref
\usepackage{graphicx}            % Allow you to include images (like graphicx). Usage: \includegraphics{path/to/file}

% Allows to set units
\usepackage[ugly]{units}        % Allows you to type units with correct spacing and font style. Usage: $\unit[100]{m}$ or $\unitfrac[100]{m}{s}$

% Additional packages
\usepackage{url}                % Lets you typeset urls. Usage: \url{http://...}
\usepackage{breakurl}           % Enables linebreaks for urls
\usepackage{xspace}             % Use \xpsace in macros to automatically insert space based on context. Usage: \newcommand{\es}{ESPResSo\xspace}
\usepackage{xcolor}             % Obviously colors. Usage: \color{red} Red text
\usepackage{booktabs}           % Nice rules for tables. Usage \begin{tabular}\toprule ... \midrule ... \bottomrule

% Source code listings
\usepackage{listings}           % Source Code Listings. Usage: \begin{lstlisting}...\end{lstlisting}
\lstloadlanguages{python}
\definecolor{lightpurple}{rgb}{0.8,0.8,1}

\lstset{
stepnumber=1,
numbersep=5pt,
numberstyle=\small\color{black},
basicstyle=\ttfamily,
%keywordstyle=\color{black},
%commentstyle=\color{black},
%stringstyle=\color{black},
frame=single,
tabsize=4,
language = python,
backgroundcolor=\color{black!5}}

\usepackage{float}

\begin{document}

\titlehead{Simulation Methods in Physics II \hfill SS 2020}
\title{Report for Worksheet 4: Charge Distribution Around a Charged Rod}
\author{Markus Baur and David Beyer}
\date{\today}
\maketitle

\tableofcontents

\section{Short Questions -- Short Answers}
\begin{itemize}
 \item \textbf{Counterion condensation}
 \item \textbf{Bjerrum length}
 \item \textbf{Mean field theory} is a quite general concept to deal with physical systems which include interactions.
\end{itemize}


\section{Analytical Solution: Poisson-Boltzmann Theory}
We used the following Python script to calculate the integration constants $\gamma$ and $R_\mathrm{M}$ and plot the solutions of the Poisson-Boltzmann equation. To obtain an equation which contains only $\gamma$, we subtracted equation (6) and (7) (in the paper). The resulting implicit equation was solved using scipy.optimize.fsolve.
\begin{lstlisting}
import numpy as np
import matplotlib.pyplot as plt
from scipy.optimize import fsolve
import time

start = time.time()

#Set plot size
width = 5.787
height = width*0.6
plt.rc('figure', figsize=(width,height))

#Use LaTeX for fonts
plt.rc('font',**{'family':'serif','serif':['Computer Modern']})
plt.rc('text', usetex=True)

#Parameters
R = 28.2
l_B = 1.0
r_0 = 1.0
r = np.linspace(1.0, R, 1000)
lambdas = [1.0, 2.0]

#Function for charge distribution
def charge_distribution(r, xi, gamma, R_M):
    ret = 1.0 - 1.0 / xi + gamma *\\
    np.tan(gamma * np.log(r / R_M)) / xi
    return ret

#Equation which has to be solved numerically
def equation1(gamma, r_0, xi, R):
    ret = gamma * np.log(r_0 / R) - np.arctan((1 - xi) / gamma)\\
    + np.arctan(1.0 / gamma)
    return ret

def equation2(gamma, R):
    ret = R / np.exp(np.arctan(1.0 / gamma) / gamma)
    return ret

#Loop over different values for lambda
for i in lambdas:
    xi = i * l_B
    #Determine parameter numerically:
    gamma = fsolve(equation1, 1.0, args=(r_0, xi, R), maxfev=10000)
    R_M = equation2(gamma, R)
    
    #Plot charge distribution
    plt.plot(r, charge_distribution(r, xi, gamma, R_M),\\
    label=r'$\lambda={}$'.format(i))
    
end = time.time()
print(end - start)

plt.xlim(1.0, R)
plt.ylim(0.0, 1.0)
plt.xscale('log')
plt.xlabel(r'$r/l_\mathrm{B}$')
plt.ylabel(r'$P(r)$')
plt.legend()
plt.tight_layout()
plt.show()
\end{lstlisting}
The resulting plot is shown in \autoref{fig:PB}, it matches the result shown on the worksheet. We can see that for the larger value of $\lambda$ (line charge density), the ions are generally closer to the rod. The calculation takes about 0.15 seconds.

\begin{figure}[h]
\centering
\includegraphics[width=\textwidth]{PB.pdf}
\caption{Plot of the Poisson-Boltzmann result for the integrated counterion density around a charged rod for $\lambda=1.0$ and $\lambda=2.0$.}
\label{fig:PB}
\end{figure}




\section{Computer Simulations}
\subsection{Mapping the Cell Model onto a Simulation}
To map the cell model onto a cubic box, we want both systems to have the same (mean) ion density. The cylindrical cell with radius $R$ and height $L$ has a volume of 
\begin{align}
 V_\mathrm{cylinder} = \pi R^2 L
\end{align}
and a total rod charge of
\begin{align}
 Q_\mathrm{cylinder,rod} = L\lambda
\end{align}
which corresponds to a mean ion density (excluding counterions) of
\begin{align}
 \langle\rho_\mathrm{cylinder, ion}\rangle = \frac{L\lambda}{\pi R^2 L} = \frac{\lambda}{\pi R^2}.
\end{align}
The cubic box has a volume of
\begin{align}
 V_\mathrm{box} = L^3
\end{align}
and a total rod charge of
\begin{align}
 Q_\mathrm{box,rod} = L\lambda
\end{align}
which corresponds to a mean ion density (excluding counterions) of
\begin{align}
 \langle\rho_\mathrm{box, ion}\rangle = \frac{L\lambda}{L^3} = \frac{\lambda}{L^2}.
\end{align}
By demanding the two densities to be identical, we obtain a box length of
\begin{align}
 L = \sqrt{\pi} R \approx 49.98.
\end{align}
For the two different values of the line charge density, this corresponds to 
\begin{align}
 N_{\mathrm{ion}} = L\lambda = \sqrt{\pi} R \lambda \approx \begin{cases}
    49.98, & \text{if $\lambda=1.0$}.\\
    99.97, & \text{if $\lambda=2.0$}.
  \end{cases}
\end{align}
ions (again excluding the counterions) of unit charge. At this point there appears one of two problems: If we allow for a non-integer charge of the ions on the rod, this leads to a non-neutral system because the counter ions have a valency of 1. If however on the other hand the ions on the rod have a unit charge, the mean line charge density is only approximately equal to $\lambda$. In the following we used the second possibility and the following number of ions on the rod with unit charge:
\begin{align}
 N_{\mathrm{ion}} =  \begin{cases}
    50, & \text{if $\lambda=1.0$}.\\
    100, & \text{if $\lambda=2.0$}.
  \end{cases}
\end{align}





\subsection{Warmup Runs}
\subsection{Equilibration and Sampling Time}
\begin{figure}[h]
\centering
\includegraphics[width=\textwidth]{energy_1.pdf}
\caption{Coulomb energy as a function of time for $\lambda=1.0$.}
\label{fig:warmup1}
\includegraphics[width=\textwidth]{energy_2.pdf}
\caption{Coulomb energy as a function of time for $\lambda=2.0$.}
\label{fig:warmup2}
\end{figure}

\subsection{Measuring the Charge Distribution}

\begin{figure}[h]
\centering
\includegraphics[width=\textwidth]{lambda_1.pdf}
\caption{Plot of the integrated counterion density calculated using Poisson-Boltzmann and MD-Simulation for $\lambda=1.0$.}
\label{fig:sim1}
\includegraphics[width=\textwidth]{lambda_2.pdf}
\caption{Plot of the integrated counterion density calculated using Poisson-Boltzmann and MD-Simulation for $\lambda=2.0$.}
\label{fig:sim2}
\end{figure}

\end{document}
