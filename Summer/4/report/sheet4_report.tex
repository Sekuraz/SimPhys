% TEMPLATE.TEX
%
% Time-stamp: <2013-03-26 11:09 olenz>
%
% This is an extensively documented LaTeX file that shows how to
% produce a good-looking document with current LaTeX (11/2012).
%
% IMPORTANT!
%
%   Some obsolete commands and packages
% ----------|-------------------------------
% obsolete  |     Replacement in LATEX 2ε
% ----------|-------------------------------
%           | local            global/switch
% ----------|-------------------------------
% {\bf ...} | \textbf{...}     \bfseries
%     -     | \emph{...}       \em
% {\it ...} | \textit{...}     \itshape
%     -     | \textmd{...}     \mdseries
% {\rm ...} | \textrm{...}     \rmfamily
% {\sc ...} | \textsc{...}     \scshape
% {\sf ...} | \textsf{...}     \sffamily
% {\sl ...} | \textsl{...}     \slshape
% {\tt ...} | \texttt{...}     \ttfamily
%     -     | \textup{...}     \upshape
%
% DON'T USE \\ TO MAKE LINEBREAKS, INSTEAD JUST LEAVE A BLANK LINE!
%
\RequirePackage[l2tabu,orthodox]{nag} % turn on warnings because of bad style
\documentclass[a4paper,10pt,bibtotoc]{scrartcl}
%
\usepackage[bottom=3.5cm, top=2cm, left=20mm,right=20mm]{geometry}
%%%%%%%%%%%%%%%%%%%%%%%%%%%%%%%%%%%%
% KOMA CLASSES
%%%%%%%%%%%%%%%%%%%%%%%%%%%%%%%%%%%%
%
% The class "scrartcl" is one of the so-called KOMA-classes, a set of
% very well done LaTeX-classes that produce a very European layout
% (e.g. titles with a sans-serif font).
%
% The KOMA classes have extensive documentation that you can access
% via the commands:
%   texdoc scrguide # in German
%   texdoc scrguien # in English
%
%
% The available classes are:
%
% scrartcl - for "articles", typically for up to ~20 pages, the
%            highest level sectioning command is \section
%
% scrreprt - for "reports", typically for up to ~200 pages, the
%            highest level sectioning command is \chapter
%
% scrbook  - for "books", for more than 200 pages, the highest level
%            sectioning command is \part.
%
% USEFUL OPTIONS
%
% a4paper  - Use a4 paper instead of the default american letter
%            format.
%
% 11pt, 12pt, 10pt
%          - Use a font with the given size.
%
% bibtotoc - Add the bibliography to the table of contents
%
% The KOMA-script classes have plenty of options to modify

% This allows to type UTF-8 characters like ä,ö,ü,ß
\usepackage[utf8]{inputenc}

\usepackage[T1]{fontenc}        % Tries to use Postscript Type 1 Fonts for better rendering
\usepackage{lmodern}            % Provides the Latin Modern Font which offers more glyphs than the default Computer Modern
\usepackage[intlimits]{amsmath} % Provides all mathematical commands

\usepackage{hyperref}           % Provides clickable links in the PDF-document for \ref
\usepackage{graphicx}            % Allow you to include images (like graphicx). Usage: \includegraphics{path/to/file}

% Allows to set units
\usepackage[ugly]{units}        % Allows you to type units with correct spacing and font style. Usage: $\unit[100]{m}$ or $\unitfrac[100]{m}{s}$

% Additional packages
\usepackage{url}                % Lets you typeset urls. Usage: \url{http://...}
\usepackage{breakurl}           % Enables linebreaks for urls
\usepackage{xspace}             % Use \xpsace in macros to automatically insert space based on context. Usage: \newcommand{\es}{ESPResSo\xspace}
\usepackage{xcolor}             % Obviously colors. Usage: \color{red} Red text
\usepackage{booktabs}           % Nice rules for tables. Usage \begin{tabular}\toprule ... \midrule ... \bottomrule

% Source code listings
\usepackage{listings}           % Source Code Listings. Usage: \begin{lstlisting}...\end{lstlisting}
\lstloadlanguages{python}
\definecolor{lightpurple}{rgb}{0.8,0.8,1}

\lstset{
stepnumber=1,
numbersep=5pt,
numberstyle=\small\color{black},
basicstyle=\ttfamily,
%keywordstyle=\color{black},
%commentstyle=\color{black},
%stringstyle=\color{black},
frame=single,
tabsize=4,
language = python,
backgroundcolor=\color{black!5}}

\usepackage{float}
\usepackage{subcaption}

\begin{document}

\titlehead{Simulation Methods in Physics II \hfill SS 2020}
\title{Report for Worksheet 4: Charge Distribution Around a Charged Rod}
\author{Markus Baur and David Beyer}
\date{\today}
\maketitle

\tableofcontents

\section{Short Questions -- Short Answers}
\begin{itemize}
 \item When looking at a charged object (like a rod) in a solution containing counterions there are two things that can happen: If entropy dominates, the counterions will try to explore all of phase space and hence not localize near the object. If however energy dominates, the counterions will localize near/onto the object, this is known as \textbf{counterion condensation}.
 \item The \textbf{Bjerrum length} $l_\mathrm{B}$ is defined as the separation for which the electrostatic coulomb energy of two elementary charges $e$ is equal to the thermal energy $k_\mathrm{B}T$:
\begin{align}
&k_\mathrm{B}T = \frac{e^2}{4\pi\epsilon_0\epsilon_r l_\mathrm{B}}\\
&\Rightarrow l_\mathrm{B} = \frac{e^2}{4\pi\epsilon_0\epsilon_r k_\mathrm{B}T}
\end{align}

 \item \textbf{Mean field theory} is a quite general concept to deal with physical systems which include interactions. The idea of mean field theory is to describe an interacting system as an effective non-interacting system where a single particle interacts with an effective external mean field which models the interaction with all the other particles. A characteristic feature of mean field theory is that the mean field has to be determined self-consistently. In general, mean field theory tends to be more accurate in higher spatial dimensions. The classical example of a mean field theory is the Weiss mean field theory which is used to describe the ferromagnetic phase transition.
\end{itemize}


\section{Analytical Solution: Poisson-Boltzmann Theory}
We used the following Python script (Poisson-Boltzmann.py) to calculate the integration constants $\gamma$ and $R_\mathrm{M}$ and plot the solutions of the Poisson-Boltzmann equation. To obtain an equation which contains only $\gamma$, we subtracted equation (6) and (7) (in the paper). The resulting implicit equation was solved using scipy.optimize.fsolve.
\begin{lstlisting}
import numpy as np
import matplotlib.pyplot as plt
from scipy.optimize import fsolve
import time

start = time.time()

#Set plot size
width = 5.787
height = width*0.6
plt.rc('figure', figsize=(width,height))

#Use LaTeX for fonts
plt.rc('font',**{'family':'serif','serif':['Computer Modern']})
plt.rc('text', usetex=True)

#Parameters
R = 28.2
l_B = 1.0
r_0 = 1.0
r = np.linspace(1.0, R, 1000)
lambdas = [1.0, 2.0]

#Function for charge distribution
def charge_distribution(r, xi, gamma, R_M):
    ret = 1.0 - 1.0 / xi + gamma *\\
    np.tan(gamma * np.log(r / R_M)) / xi
    return ret

#Equation which has to be solved numerically
def equation1(gamma, r_0, xi, R):
    ret = gamma * np.log(r_0 / R) - np.arctan((1 - xi) / gamma)\\
    + np.arctan(1.0 / gamma)
    return ret

def equation2(gamma, R):
    ret = R / np.exp(np.arctan(1.0 / gamma) / gamma)
    return ret

#Loop over different values for lambda
for i in lambdas:
    xi = i * l_B
    #Determine parameter numerically:
    gamma = fsolve(equation1, 1.0, args=(r_0, xi, R), maxfev=10000)
    R_M = equation2(gamma, R)
    
    #Plot charge distribution
    plt.plot(r, charge_distribution(r, xi, gamma, R_M),\\
    label=r'$\lambda={}$'.format(i))
    
end = time.time()
print(end - start)

plt.xlim(1.0, R)
plt.ylim(0.0, 1.0)
plt.xscale('log')
plt.xlabel(r'$r/l_\mathrm{B}$')
plt.ylabel(r'$P(r)$')
plt.legend()
plt.tight_layout()
plt.show()
\end{lstlisting}
The resulting plot is shown in \autoref{fig:PB}, it matches the result shown on the worksheet. We can see that for the larger value of $\lambda$ (line charge density), the counterions are generally closer to the rod. The calculation takes about 0.15 seconds.

\begin{figure}[h]
\centering
\includegraphics[width=0.7\textwidth]{PB.pdf}
\caption{Plot of the Poisson-Boltzmann result for the integrated counterion density around a charged rod for $\lambda=1.0$ and $\lambda=2.0$.}
\label{fig:PB}
\end{figure}




\section{Computer Simulations}
\subsection{Mapping the Cell Model onto a Simulation}
To map the cell model onto a cubic box, we want both systems to have the same (mean) ion density. The cylindrical cell with radius $R$ and height $L$ has a volume of 
\begin{align}
 V_\mathrm{cylinder} = \pi R^2 L
\end{align}
and a total rod charge of
\begin{align}
 Q_\mathrm{cylinder,rod} = L\lambda
\end{align}
which corresponds to a mean ion density (excluding counterions) of
\begin{align}
 \langle\rho_\mathrm{cylinder, ion}\rangle = \frac{L\lambda}{\pi R^2 L} = \frac{\lambda}{\pi R^2}.
\end{align}
The cubic box has a volume of
\begin{align}
 V_\mathrm{box} = L^3
\end{align}
and a total rod charge of
\begin{align}
 Q_\mathrm{box,rod} = L\lambda
\end{align}
which corresponds to a mean ion density (excluding counterions) of
\begin{align}
 \langle\rho_\mathrm{box, ion}\rangle = \frac{L\lambda}{L^3} = \frac{\lambda}{L^2}.
\end{align}
By demanding the two densities to be identical, we obtain a box length of
\begin{align}
 L = \sqrt{\pi} R \approx 49.98.
\end{align}
For the two different values of the line charge density, this corresponds to 
\begin{align}
 N_{\mathrm{ion}} = L\lambda = \sqrt{\pi} R \lambda \approx \begin{cases}
    49.98, & \text{if $\lambda=1.0$}.\\
    99.97, & \text{if $\lambda=2.0$}.
  \end{cases}
\end{align}
ions (again excluding the counterions) of unit charge. At this point there appears one of two problems: If we allow for a non-integer charge of the ions on the rod, this leads to a non-neutral system because the counter ions have a valency of 1. If however on the other hand the ions on the rod have a unit charge, the mean line charge density is only approximately equal to $\lambda$. In the following we chose to have a neutral system and used the following number of ions on the rod with unit charge:
\begin{align}
 N_{\mathrm{ion}} =  \begin{cases}
    50, & \text{if $\lambda=1.0$}.\\
    100, & \text{if $\lambda=2.0$}.
  \end{cases}
\end{align}
Due to the neutrality condition, the number of counterions is the same.
In the provided script we had to set $L$:
\begin{lstlisting}
L = np.sqrt(np.pi) * 28.2
\end{lstlisting}
To calculate the correct number of ions and counterions we used
\begin{lstlisting}
num_rod_beads = int(round(line_dens * L))
num_ci = int(round(total_rod_charge/valency_ci))
\end{lstlisting}





\subsection{Warmup Runs}
First we ran a test run for both line charge densities and visualized the results in VMD. To fold all particles back into the first unit cell, we used the following commands in the Tk-console:
\begin{lstlisting}
 pbc set {49.9832 49.9832 49.9832} -all
 pbc box -off
 pbc wrap -all
\end{lstlisting}
\autoref{fig:VMD} shows snapshots of the two systems at $t=0$ and $t=1000$. We can see that there is a crowding of ions near the rod for both systems after some time has elapsed, however this effect is much more pronounced for $\lambda=2.0$. This observation is in agreement with the analytical results from the Poisson-Boltzmann mean field theory which were plotted in the previous section.

\begin{figure}[H]
\begin{subfigure}{.5\textwidth}
  \centering
  % include first image
  \includegraphics[width=\linewidth]{initial_1.png}  
  \caption{Snapshot of the system with line charge density $\lambda=1.0$ at time $t=0$.}
\end{subfigure}
\begin{subfigure}{.5\textwidth}
  \centering
  % include second image
  \includegraphics[width=\linewidth]{final_1.png}  
  \caption{Snapshot of the system with line charge density $\lambda=1.0$ at time $t=1000$.}
\end{subfigure}
\begin{subfigure}{.5\textwidth}
  \centering
  % include first image
  \includegraphics[width=\linewidth]{initial_2.png}  
  \caption{Snapshot of the system with line charge density $\lambda=2.0$ at time $t=0$.}
\end{subfigure}
\begin{subfigure}{.5\textwidth}
  \centering
  % include second image
  \includegraphics[width=\linewidth]{final_2.png}  
  \caption{Snapshot of the system with line charge density $\lambda=2.0$ at time $t=1000$.}
\end{subfigure}
\caption{VMD-snapshots of the systems with line charge densities $\lambda=1.0$ and $\lambda=2.0$.}
\label{fig:VMD}
\end{figure}

\subsection{Equilibration and Sampling Time}
\autoref{equi1} and \autoref{equi2} show the Coulomb energy as a function of time for the different values of $\lambda$. The plots suggest that the equilibration time is about 100 time units (=frames). To be on the safe side, we only used data after the first 200 frames.

\autoref{equi3} and \autoref{equi4} show the Coulomb energy as a function of time for the different values of $\lambda$ and a longer time interval.
The plots suggest that the slowest fluctuations take about 100 frames.
To get statistically meaningful results, we ran simulations of 10000 frames for both systems.

\begin{figure}[ht]
\begin{subfigure}{.5\textwidth}
  \centering
  % include first image
  \includegraphics[width=\linewidth]{energy_1.pdf}  
  \caption{Coulomb energy as a function of time for $\lambda=1.0$.}
  \label{equi1}
\end{subfigure}
\begin{subfigure}{.5\textwidth}
  \centering
  % include second image
  \includegraphics[width=\linewidth]{energy_2.pdf}  
  \caption{Coulomb energy as a function of time for $\lambda=2.0$.}
  \label{equi2}
\end{subfigure}
\begin{subfigure}{.5\textwidth}
  \centering
  % include first image
  \includegraphics[width=\linewidth]{energy_1_long.pdf}  
  \caption{Coulomb energy as a function of time for $\lambda=1.0$.}
  \label{equi3}
\end{subfigure}
\begin{subfigure}{.5\textwidth}
  \centering
  % include second image
  \includegraphics[width=\linewidth]{energy_2_long.pdf}  
  \caption{Coulomb energy as a function of time for $\lambda=2.0$.}
  \label{equi4}
\end{subfigure}
\label{fig:fig}
\end{figure}

\subsection{Measuring the Charge Distribution}
To measure the charge distribution, we have to write the particle coordinates to an output file, this can be done in the following way:
\begin{lstlisting}
for p in system.part:
        if p.type==ci_type:
            positions_file.write("{}  \t {} \n"\\
            .format(p.pos_folded[0], p.pos_folded[1]))
\end{lstlisting}
We use the option pos\_folded to fold the coordinates back into the first unit cell. 
Because we are only interested in the direction orthogonal to the rod, we do not need the $z$ coordinates.

To calculate the integrated counterion density, we wrote the script histogram.py (see below). 
First, we generate an array of orthogonal distances from the rod, this is done by subtracting $L/2$ from all particles coordinates (the rod is positioned in the center) and applying the NumPy function linalg.norm.
Next, we generate a histogram of the distances using the NumPy function histogram. 
To finally calculate the integrated counterion density, we have to sum all particles in the histogram up to a distance $r$, this is achieved by using the NumPy function cumsum.
We then plot the resulting (normalized) integrated charge density and also plot the analytical (Poisson-Boltzmann) result for comparison.
\begin{lstlisting}
import numpy as np
import matplotlib.pyplot as plt
import os
from scipy.optimize import fsolve


dir_path = os.path.dirname(os.path.realpath(__file__))
dir_file = dir_path + '/positions_2.dat'
data = np.loadtxt(dir_file,unpack=False) - np.sqrt(np.pi)\\
* 28.2/2.0

#Set plot size
width = 5.787
height = width*0.6
plt.rc('figure', figsize=(width,height))

#Use LaTeX for fonts
plt.rc('font',**{'family':'serif','serif':['Computer Modern']})
plt.rc('text', usetex=True)

r = np.linalg.norm(data, axis=1)
hist, bin_edges = np.histogram(r[20000:], bins\\
= np.linspace(1.0, 28.2, endpoint=True, num=1000))

real_hist = np.cumsum(hist)
real_hist = real_hist / np.amax(real_hist)

#Parameters
R = 28.2
l_B = 1.0
r_0 = 1.0
r = np.linspace(1.0, R, 1000)
lambdas = [2.0]

#Function for charge distribution
def charge_distribution(r, xi, gamma, R_M):
    ret = 1.0 - 1.0 / xi + gamma\\
    * np.tan(gamma * np.log(r / R_M)) / xi
    return ret

#Equation which has to be solved numerically
def equation1(gamma, r_0, xi, R):
    ret = gamma * np.log(r_0 / R)\\
    - np.arctan((1 - xi) / gamma) + np.arctan(1.0 / gamma)
    return ret

def equation2(gamma, R):
    ret = R / np.exp(np.arctan(1.0 / gamma) / gamma)
    return ret

#Loop over different values for lambda
for i in lambdas:
    xi = i * l_B
    #Determine parameter numerically:
    gamma = fsolve(equation1, 1.0, args=(r_0, xi, R), maxfev=10000)
    R_M = equation2(gamma, R)
    
    #Plot charge distribution
    plt.plot(r, charge_distribution(r, xi, gamma, R_M),\\
    label=r'Poisson-Boltzmann')

plt.xlim(1.0, R)
plt.ylim(0.0, 1.0)

plt.plot(bin_edges[:-1], real_hist, label=r'Simulation')
plt.xscale('log')
plt.xlabel(r'$r/l_\mathrm{B}$')
plt.ylabel(r'$P(r)$')
plt.legend()
plt.tight_layout()
plt.show()
\end{lstlisting}
\autoref{fig:counter1} and \autoref{fig:counter2} show the calculated integrated counterion densities.
We can see that the simulation results are quite close the prediction of the Poisson-Boltzmann mean field theory. For both values of $\lambda$, $P(r)$ from the simulation is slightly larger then the analytical result, except for large values of $r$. This deviation for large $r$, which is shown in detail in \autoref{fig:counter3} and \autoref{fig:counter4} can be explained by the fact that simulation box is cubic while the cell model is cylindrical: in the cylinder, the maximum distance from the rod is $R$, while in the box the maximum distance is $L/\sqrt{2}>R$. This means that the counterions in the simulation can reach $r$ values which are forbidden in the cell model and hence $P(r)$ for the simulation gets flattened for large $r$. The fact that $P(r)$ from the simulations is slightly larger then the analytical result and does not go to zero for $r=1.0$ (see \autoref{fig:counter5} and \autoref{fig:counter6}) may be explained by the fact that the counterions in the simulation can (partially) overlap with the stationary ions of the rod.

\begin{figure}[ht]
\begin{subfigure}{.5\textwidth}
  \centering
  % include first image
  \includegraphics[width=\linewidth]{lambda_1.pdf}  
  \caption{Integrated counterion density for $\lambda=1.0$.}
  \label{fig:counter1}
\end{subfigure}
\begin{subfigure}{.5\textwidth}
  \centering
  % include second image
  \includegraphics[width=\linewidth]{lambda_2.pdf}  
  \caption{Integrated counterion density for $\lambda=2.0$.}
  \label{fig:counter2}
\end{subfigure}
\begin{subfigure}{.5\textwidth}
  \centering
  % include first image
  \includegraphics[width=\linewidth]{lambda_1_detail_1.pdf}  
  \caption{Detail of the integrated counterion density for large values of $r$ and $\lambda=1.0$.}
  \label{fig:counter3}
\end{subfigure}
\begin{subfigure}{.5\textwidth}
  \centering
  % include second image
  \includegraphics[width=\linewidth]{lambda_2_detail_2.pdf}  
  \caption{Detail of the integrated counterion density for large values of $r$ and $\lambda=2.0$.}
  \label{fig:counter4}
\end{subfigure}
\begin{subfigure}{.5\textwidth}
  \centering
  % include first image
  \includegraphics[width=\linewidth]{lambda_1_detail.pdf}  
  \caption{Detail of the integrated counterion density for small values of $r$ and $\lambda=1.0$.}
  \label{fig:counter5}
\end{subfigure}
\begin{subfigure}{.5\textwidth}
  \centering
  % include second image
  \includegraphics[width=\linewidth]{lambda_2_detail.pdf}  
  \caption{Detail of the integrated counterion density for small values of $r$ and $\lambda=2.0$.}
  \label{fig:counter6}
\end{subfigure}
\caption{Integrated counterion densities from the simulations and Poisson-Boltzmann theory.}
\label{fig:fig}
\end{figure}

\end{document}
