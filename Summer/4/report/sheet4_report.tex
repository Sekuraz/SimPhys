% TEMPLATE.TEX
%
% Time-stamp: <2013-03-26 11:09 olenz>
%
% This is an extensively documented LaTeX file that shows how to
% produce a good-looking document with current LaTeX (11/2012).
%
% IMPORTANT!
%
%   Some obsolete commands and packages
% ----------|-------------------------------
% obsolete  |     Replacement in LATEX 2ε
% ----------|-------------------------------
%           | local            global/switch
% ----------|-------------------------------
% {\bf ...} | \textbf{...}     \bfseries
%     -     | \emph{...}       \em
% {\it ...} | \textit{...}     \itshape
%     -     | \textmd{...}     \mdseries
% {\rm ...} | \textrm{...}     \rmfamily
% {\sc ...} | \textsc{...}     \scshape
% {\sf ...} | \textsf{...}     \sffamily
% {\sl ...} | \textsl{...}     \slshape
% {\tt ...} | \texttt{...}     \ttfamily
%     -     | \textup{...}     \upshape
%
% DON'T USE \\ TO MAKE LINEBREAKS, INSTEAD JUST LEAVE A BLANK LINE!
%
\RequirePackage[l2tabu,orthodox]{nag} % turn on warnings because of bad style
\documentclass[a4paper,10pt,bibtotoc]{scrartcl}
%
\usepackage[bottom=3.5cm, top=2cm]{geometry}
%%%%%%%%%%%%%%%%%%%%%%%%%%%%%%%%%%%%
% KOMA CLASSES
%%%%%%%%%%%%%%%%%%%%%%%%%%%%%%%%%%%%
%
% The class "scrartcl" is one of the so-called KOMA-classes, a set of
% very well done LaTeX-classes that produce a very European layout
% (e.g. titles with a sans-serif font).
%
% The KOMA classes have extensive documentation that you can access
% via the commands:
%   texdoc scrguide # in German
%   texdoc scrguien # in English
%
%
% The available classes are:
%
% scrartcl - for "articles", typically for up to ~20 pages, the
%            highest level sectioning command is \section
%
% scrreprt - for "reports", typically for up to ~200 pages, the
%            highest level sectioning command is \chapter
%
% scrbook  - for "books", for more than 200 pages, the highest level
%            sectioning command is \part.
%
% USEFUL OPTIONS
%
% a4paper  - Use a4 paper instead of the default american letter
%            format.
%
% 11pt, 12pt, 10pt
%          - Use a font with the given size.
%
% bibtotoc - Add the bibliography to the table of contents
%
% The KOMA-script classes have plenty of options to modify

% This allows to type UTF-8 characters like ä,ö,ü,ß
\usepackage[utf8]{inputenc}

\usepackage[T1]{fontenc}        % Tries to use Postscript Type 1 Fonts for better rendering
\usepackage{lmodern}            % Provides the Latin Modern Font which offers more glyphs than the default Computer Modern
\usepackage[intlimits]{amsmath} % Provides all mathematical commands

\usepackage{hyperref}           % Provides clickable links in the PDF-document for \ref
\usepackage{graphicx}            % Allow you to include images (like graphicx). Usage: \includegraphics{path/to/file}

% Allows to set units
\usepackage[ugly]{units}        % Allows you to type units with correct spacing and font style. Usage: $\unit[100]{m}$ or $\unitfrac[100]{m}{s}$

% Additional packages
\usepackage{url}                % Lets you typeset urls. Usage: \url{http://...}
\usepackage{breakurl}           % Enables linebreaks for urls
\usepackage{xspace}             % Use \xpsace in macros to automatically insert space based on context. Usage: \newcommand{\es}{ESPResSo\xspace}
\usepackage{xcolor}             % Obviously colors. Usage: \color{red} Red text
\usepackage{booktabs}           % Nice rules for tables. Usage \begin{tabular}\toprule ... \midrule ... \bottomrule

% Source code listings
\usepackage{listings}           % Source Code Listings. Usage: \begin{lstlisting}...\end{lstlisting}
\lstloadlanguages{python}
\definecolor{lightpurple}{rgb}{0.8,0.8,1}

\lstset{
stepnumber=1,
numbersep=5pt,
numberstyle=\small\color{black},
basicstyle=\ttfamily,
%keywordstyle=\color{black},
%commentstyle=\color{black},
%stringstyle=\color{black},
frame=single,
tabsize=4,
language = python,
backgroundcolor=\color{black!5}}

\usepackage{float}

\begin{document}

\titlehead{Simulation Methods in Physics II \hfill SS 2020}
\title{Report for Worksheet 4: Charge Distribution Around a Charged Rod}
\author{Markus Baur and David Beyer}
\date{\today}
\maketitle

\tableofcontents

\section{Short Questions -- Short Answers}
\begin{itemize}
 \item \textbf{Mean field theory} is a quite general concept to deal with physical systems which include interactions.
\end{itemize}


\section{Analytical Solution: Poisson-Boltzmann Theory}
\begin{lstlisting}
import numpy as np
import matplotlib.pyplot as plt
from scipy.optimize import fsolve

#Set plot size
width = 5.787
height = width*0.6
plt.rc('figure', figsize=(width,height))

#Use LaTeX for fonts
plt.rc('font',**{'family':'serif','serif':['Computer Modern']})
plt.rc('text', usetex=True)

#Parameters
R = 28.2
l_B = 1.0
r_0 = 1.0
r = np.linspace(1.0, R, 1000)
lambdas = [1.0, 2.0]

#Function for charge distribution
def charge_distribution(r, xi, gamma, R_M):
    ret = 1.0 - 1.0 / xi + gamma * np.tan(gamma *\\
    np.log(r / R_M)) / xi
    return ret

#Equation which hase to be solved numerically
def equation1(gamma, r_0, xi, R):
    ret = gamma * np.log(r_0 / R) - np.arctan((1 - xi) / gamma)\\
    + np.arctan(1.0 / gamma)
    return ret

def equation2(gamma, R):
    ret = R / np.exp(np.arctan(1.0 / gamma) / gamma)
    return ret

#Loop over different values for lambda
for i in lambdas:
    xi = i * l_B
    #Determine parameter numerically:
    gamma = fsolve(equation1, 1.0, args=(r_0, xi, R),\\
    maxfev=10000)
    R_M = equation2(gamma, R)
    
    #Plot charge distribution
    plt.plot(r, charge_distribution(r, xi, gamma, R_M),\\
    label=r'$\lambda={}$'.format(i))

plt.xlim(1.0, R)
plt.ylim(0.0, 1.0)
plt.xscale('log')
plt.xlabel(r'$r/l_\mathrm{B}$')
plt.ylabel(r'$P(r)$')
plt.legend()
plt.tight_layout()
plt.show()
\end{lstlisting}


\section{Computer Simulations}
\subsection{Mapping the Cell Model onto a Simulation}
\subsection{Warmup Runs}
\subsection{Equilibration and Sampling Time}
\subsection{Measuring the Charge Distribution}

\begin{thebibliography}{9}
		
		\bibitem{rubinstein}
		\textsc{M. Rubinstein} and \textsc{R. H. Colby},
		\emph{Polymer Physics},
		Oxford University Press,
		Oxford,
		2003.


\end{thebibliography}

\end{document}
